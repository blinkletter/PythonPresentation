\documentclass[11pt]{article}

    \usepackage[breakable]{tcolorbox}
    \usepackage{parskip} % Stop auto-indenting (to mimic markdown behaviour)
    

    % Basic figure setup, for now with no caption control since it's done
    % automatically by Pandoc (which extracts ![](path) syntax from Markdown).
    \usepackage{graphicx}
    % Keep aspect ratio if custom image width or height is specified
    \setkeys{Gin}{keepaspectratio}
    % Maintain compatibility with old templates. Remove in nbconvert 6.0
    \let\Oldincludegraphics\includegraphics
    % Ensure that by default, figures have no caption (until we provide a
    % proper Figure object with a Caption API and a way to capture that
    % in the conversion process - todo).
    \usepackage{caption}
    \DeclareCaptionFormat{nocaption}{}
    \captionsetup{format=nocaption,aboveskip=0pt,belowskip=0pt}

    \usepackage{float}
    \floatplacement{figure}{H} % forces figures to be placed at the correct location
    \usepackage{xcolor} % Allow colors to be defined
    \usepackage{enumerate} % Needed for markdown enumerations to work
    \usepackage{geometry} % Used to adjust the document margins
    \usepackage{amsmath} % Equations
    \usepackage{amssymb} % Equations
    \usepackage{textcomp} % defines textquotesingle
    % Hack from http://tex.stackexchange.com/a/47451/13684:
    \AtBeginDocument{%
        \def\PYZsq{\textquotesingle}% Upright quotes in Pygmentized code
    }
    \usepackage{upquote} % Upright quotes for verbatim code
    \usepackage{eurosym} % defines \euro

    \usepackage{iftex}
    \ifPDFTeX
        \usepackage[T1]{fontenc}
        \IfFileExists{alphabeta.sty}{
              \usepackage{alphabeta}
          }{
              \usepackage[mathletters]{ucs}
              \usepackage[utf8x]{inputenc}
          }
    \else
        \usepackage{fontspec}
        \usepackage{unicode-math}
    \fi

    \usepackage{fancyvrb} % verbatim replacement that allows latex
    \usepackage{grffile} % extends the file name processing of package graphics
                         % to support a larger range
    \makeatletter % fix for old versions of grffile with XeLaTeX
    \@ifpackagelater{grffile}{2019/11/01}
    {
      % Do nothing on new versions
    }
    {
      \def\Gread@@xetex#1{%
        \IfFileExists{"\Gin@base".bb}%
        {\Gread@eps{\Gin@base.bb}}%
        {\Gread@@xetex@aux#1}%
      }
    }
    \makeatother
    \usepackage[Export]{adjustbox} % Used to constrain images to a maximum size
    \adjustboxset{max size={0.9\linewidth}{0.9\paperheight}}

    % The hyperref package gives us a pdf with properly built
    % internal navigation ('pdf bookmarks' for the table of contents,
    % internal cross-reference links, web links for URLs, etc.)
    \usepackage{hyperref}
    % The default LaTeX title has an obnoxious amount of whitespace. By default,
    % titling removes some of it. It also provides customization options.
    \usepackage{titling}
    \usepackage{longtable} % longtable support required by pandoc >1.10
    \usepackage{booktabs}  % table support for pandoc > 1.12.2
    \usepackage{array}     % table support for pandoc >= 2.11.3
    \usepackage{calc}      % table minipage width calculation for pandoc >= 2.11.1
    \usepackage[inline]{enumitem} % IRkernel/repr support (it uses the enumerate* environment)
    \usepackage[normalem]{ulem} % ulem is needed to support strikethroughs (\sout)
                                % normalem makes italics be italics, not underlines
    \usepackage{soul}      % strikethrough (\st) support for pandoc >= 3.0.0
    \usepackage{mathrsfs}
    

    
    % Colors for the hyperref package
    \definecolor{urlcolor}{rgb}{0,.145,.698}
    \definecolor{linkcolor}{rgb}{.71,0.21,0.01}
    \definecolor{citecolor}{rgb}{.12,.54,.11}

    % ANSI colors
    \definecolor{ansi-black}{HTML}{3E424D}
    \definecolor{ansi-black-intense}{HTML}{282C36}
    \definecolor{ansi-red}{HTML}{E75C58}
    \definecolor{ansi-red-intense}{HTML}{B22B31}
    \definecolor{ansi-green}{HTML}{00A250}
    \definecolor{ansi-green-intense}{HTML}{007427}
    \definecolor{ansi-yellow}{HTML}{DDB62B}
    \definecolor{ansi-yellow-intense}{HTML}{B27D12}
    \definecolor{ansi-blue}{HTML}{208FFB}
    \definecolor{ansi-blue-intense}{HTML}{0065CA}
    \definecolor{ansi-magenta}{HTML}{D160C4}
    \definecolor{ansi-magenta-intense}{HTML}{A03196}
    \definecolor{ansi-cyan}{HTML}{60C6C8}
    \definecolor{ansi-cyan-intense}{HTML}{258F8F}
    \definecolor{ansi-white}{HTML}{C5C1B4}
    \definecolor{ansi-white-intense}{HTML}{A1A6B2}
    \definecolor{ansi-default-inverse-fg}{HTML}{FFFFFF}
    \definecolor{ansi-default-inverse-bg}{HTML}{000000}

    % common color for the border for error outputs.
    \definecolor{outerrorbackground}{HTML}{FFDFDF}

    % commands and environments needed by pandoc snippets
    % extracted from the output of `pandoc -s`
    \providecommand{\tightlist}{%
      \setlength{\itemsep}{0pt}\setlength{\parskip}{0pt}}
    \DefineVerbatimEnvironment{Highlighting}{Verbatim}{commandchars=\\\{\}}
    % Add ',fontsize=\small' for more characters per line
    \newenvironment{Shaded}{}{}
    \newcommand{\KeywordTok}[1]{\textcolor[rgb]{0.00,0.44,0.13}{\textbf{{#1}}}}
    \newcommand{\DataTypeTok}[1]{\textcolor[rgb]{0.56,0.13,0.00}{{#1}}}
    \newcommand{\DecValTok}[1]{\textcolor[rgb]{0.25,0.63,0.44}{{#1}}}
    \newcommand{\BaseNTok}[1]{\textcolor[rgb]{0.25,0.63,0.44}{{#1}}}
    \newcommand{\FloatTok}[1]{\textcolor[rgb]{0.25,0.63,0.44}{{#1}}}
    \newcommand{\CharTok}[1]{\textcolor[rgb]{0.25,0.44,0.63}{{#1}}}
    \newcommand{\StringTok}[1]{\textcolor[rgb]{0.25,0.44,0.63}{{#1}}}
    \newcommand{\CommentTok}[1]{\textcolor[rgb]{0.38,0.63,0.69}{\textit{{#1}}}}
    \newcommand{\OtherTok}[1]{\textcolor[rgb]{0.00,0.44,0.13}{{#1}}}
    \newcommand{\AlertTok}[1]{\textcolor[rgb]{1.00,0.00,0.00}{\textbf{{#1}}}}
    \newcommand{\FunctionTok}[1]{\textcolor[rgb]{0.02,0.16,0.49}{{#1}}}
    \newcommand{\RegionMarkerTok}[1]{{#1}}
    \newcommand{\ErrorTok}[1]{\textcolor[rgb]{1.00,0.00,0.00}{\textbf{{#1}}}}
    \newcommand{\NormalTok}[1]{{#1}}

    % Additional commands for more recent versions of Pandoc
    \newcommand{\ConstantTok}[1]{\textcolor[rgb]{0.53,0.00,0.00}{{#1}}}
    \newcommand{\SpecialCharTok}[1]{\textcolor[rgb]{0.25,0.44,0.63}{{#1}}}
    \newcommand{\VerbatimStringTok}[1]{\textcolor[rgb]{0.25,0.44,0.63}{{#1}}}
    \newcommand{\SpecialStringTok}[1]{\textcolor[rgb]{0.73,0.40,0.53}{{#1}}}
    \newcommand{\ImportTok}[1]{{#1}}
    \newcommand{\DocumentationTok}[1]{\textcolor[rgb]{0.73,0.13,0.13}{\textit{{#1}}}}
    \newcommand{\AnnotationTok}[1]{\textcolor[rgb]{0.38,0.63,0.69}{\textbf{\textit{{#1}}}}}
    \newcommand{\CommentVarTok}[1]{\textcolor[rgb]{0.38,0.63,0.69}{\textbf{\textit{{#1}}}}}
    \newcommand{\VariableTok}[1]{\textcolor[rgb]{0.10,0.09,0.49}{{#1}}}
    \newcommand{\ControlFlowTok}[1]{\textcolor[rgb]{0.00,0.44,0.13}{\textbf{{#1}}}}
    \newcommand{\OperatorTok}[1]{\textcolor[rgb]{0.40,0.40,0.40}{{#1}}}
    \newcommand{\BuiltInTok}[1]{{#1}}
    \newcommand{\ExtensionTok}[1]{{#1}}
    \newcommand{\PreprocessorTok}[1]{\textcolor[rgb]{0.74,0.48,0.00}{{#1}}}
    \newcommand{\AttributeTok}[1]{\textcolor[rgb]{0.49,0.56,0.16}{{#1}}}
    \newcommand{\InformationTok}[1]{\textcolor[rgb]{0.38,0.63,0.69}{\textbf{\textit{{#1}}}}}
    \newcommand{\WarningTok}[1]{\textcolor[rgb]{0.38,0.63,0.69}{\textbf{\textit{{#1}}}}}


    % Define a nice break command that doesn't care if a line doesn't already
    % exist.
    \def\br{\hspace*{\fill} \\* }
    % Math Jax compatibility definitions
    \def\gt{>}
    \def\lt{<}
    \let\Oldtex\TeX
    \let\Oldlatex\LaTeX
    \renewcommand{\TeX}{\textrm{\Oldtex}}
    \renewcommand{\LaTeX}{\textrm{\Oldlatex}}
    % Document parameters
    % Document title
    \title{01X\_PlotBasics}
    
    
    
    
    
    
    
% Pygments definitions
\makeatletter
\def\PY@reset{\let\PY@it=\relax \let\PY@bf=\relax%
    \let\PY@ul=\relax \let\PY@tc=\relax%
    \let\PY@bc=\relax \let\PY@ff=\relax}
\def\PY@tok#1{\csname PY@tok@#1\endcsname}
\def\PY@toks#1+{\ifx\relax#1\empty\else%
    \PY@tok{#1}\expandafter\PY@toks\fi}
\def\PY@do#1{\PY@bc{\PY@tc{\PY@ul{%
    \PY@it{\PY@bf{\PY@ff{#1}}}}}}}
\def\PY#1#2{\PY@reset\PY@toks#1+\relax+\PY@do{#2}}

\@namedef{PY@tok@w}{\def\PY@tc##1{\textcolor[rgb]{0.73,0.73,0.73}{##1}}}
\@namedef{PY@tok@c}{\let\PY@it=\textit\def\PY@tc##1{\textcolor[rgb]{0.24,0.48,0.48}{##1}}}
\@namedef{PY@tok@cp}{\def\PY@tc##1{\textcolor[rgb]{0.61,0.40,0.00}{##1}}}
\@namedef{PY@tok@k}{\let\PY@bf=\textbf\def\PY@tc##1{\textcolor[rgb]{0.00,0.50,0.00}{##1}}}
\@namedef{PY@tok@kp}{\def\PY@tc##1{\textcolor[rgb]{0.00,0.50,0.00}{##1}}}
\@namedef{PY@tok@kt}{\def\PY@tc##1{\textcolor[rgb]{0.69,0.00,0.25}{##1}}}
\@namedef{PY@tok@o}{\def\PY@tc##1{\textcolor[rgb]{0.40,0.40,0.40}{##1}}}
\@namedef{PY@tok@ow}{\let\PY@bf=\textbf\def\PY@tc##1{\textcolor[rgb]{0.67,0.13,1.00}{##1}}}
\@namedef{PY@tok@nb}{\def\PY@tc##1{\textcolor[rgb]{0.00,0.50,0.00}{##1}}}
\@namedef{PY@tok@nf}{\def\PY@tc##1{\textcolor[rgb]{0.00,0.00,1.00}{##1}}}
\@namedef{PY@tok@nc}{\let\PY@bf=\textbf\def\PY@tc##1{\textcolor[rgb]{0.00,0.00,1.00}{##1}}}
\@namedef{PY@tok@nn}{\let\PY@bf=\textbf\def\PY@tc##1{\textcolor[rgb]{0.00,0.00,1.00}{##1}}}
\@namedef{PY@tok@ne}{\let\PY@bf=\textbf\def\PY@tc##1{\textcolor[rgb]{0.80,0.25,0.22}{##1}}}
\@namedef{PY@tok@nv}{\def\PY@tc##1{\textcolor[rgb]{0.10,0.09,0.49}{##1}}}
\@namedef{PY@tok@no}{\def\PY@tc##1{\textcolor[rgb]{0.53,0.00,0.00}{##1}}}
\@namedef{PY@tok@nl}{\def\PY@tc##1{\textcolor[rgb]{0.46,0.46,0.00}{##1}}}
\@namedef{PY@tok@ni}{\let\PY@bf=\textbf\def\PY@tc##1{\textcolor[rgb]{0.44,0.44,0.44}{##1}}}
\@namedef{PY@tok@na}{\def\PY@tc##1{\textcolor[rgb]{0.41,0.47,0.13}{##1}}}
\@namedef{PY@tok@nt}{\let\PY@bf=\textbf\def\PY@tc##1{\textcolor[rgb]{0.00,0.50,0.00}{##1}}}
\@namedef{PY@tok@nd}{\def\PY@tc##1{\textcolor[rgb]{0.67,0.13,1.00}{##1}}}
\@namedef{PY@tok@s}{\def\PY@tc##1{\textcolor[rgb]{0.73,0.13,0.13}{##1}}}
\@namedef{PY@tok@sd}{\let\PY@it=\textit\def\PY@tc##1{\textcolor[rgb]{0.73,0.13,0.13}{##1}}}
\@namedef{PY@tok@si}{\let\PY@bf=\textbf\def\PY@tc##1{\textcolor[rgb]{0.64,0.35,0.47}{##1}}}
\@namedef{PY@tok@se}{\let\PY@bf=\textbf\def\PY@tc##1{\textcolor[rgb]{0.67,0.36,0.12}{##1}}}
\@namedef{PY@tok@sr}{\def\PY@tc##1{\textcolor[rgb]{0.64,0.35,0.47}{##1}}}
\@namedef{PY@tok@ss}{\def\PY@tc##1{\textcolor[rgb]{0.10,0.09,0.49}{##1}}}
\@namedef{PY@tok@sx}{\def\PY@tc##1{\textcolor[rgb]{0.00,0.50,0.00}{##1}}}
\@namedef{PY@tok@m}{\def\PY@tc##1{\textcolor[rgb]{0.40,0.40,0.40}{##1}}}
\@namedef{PY@tok@gh}{\let\PY@bf=\textbf\def\PY@tc##1{\textcolor[rgb]{0.00,0.00,0.50}{##1}}}
\@namedef{PY@tok@gu}{\let\PY@bf=\textbf\def\PY@tc##1{\textcolor[rgb]{0.50,0.00,0.50}{##1}}}
\@namedef{PY@tok@gd}{\def\PY@tc##1{\textcolor[rgb]{0.63,0.00,0.00}{##1}}}
\@namedef{PY@tok@gi}{\def\PY@tc##1{\textcolor[rgb]{0.00,0.52,0.00}{##1}}}
\@namedef{PY@tok@gr}{\def\PY@tc##1{\textcolor[rgb]{0.89,0.00,0.00}{##1}}}
\@namedef{PY@tok@ge}{\let\PY@it=\textit}
\@namedef{PY@tok@gs}{\let\PY@bf=\textbf}
\@namedef{PY@tok@ges}{\let\PY@bf=\textbf\let\PY@it=\textit}
\@namedef{PY@tok@gp}{\let\PY@bf=\textbf\def\PY@tc##1{\textcolor[rgb]{0.00,0.00,0.50}{##1}}}
\@namedef{PY@tok@go}{\def\PY@tc##1{\textcolor[rgb]{0.44,0.44,0.44}{##1}}}
\@namedef{PY@tok@gt}{\def\PY@tc##1{\textcolor[rgb]{0.00,0.27,0.87}{##1}}}
\@namedef{PY@tok@err}{\def\PY@bc##1{{\setlength{\fboxsep}{\string -\fboxrule}\fcolorbox[rgb]{1.00,0.00,0.00}{1,1,1}{\strut ##1}}}}
\@namedef{PY@tok@kc}{\let\PY@bf=\textbf\def\PY@tc##1{\textcolor[rgb]{0.00,0.50,0.00}{##1}}}
\@namedef{PY@tok@kd}{\let\PY@bf=\textbf\def\PY@tc##1{\textcolor[rgb]{0.00,0.50,0.00}{##1}}}
\@namedef{PY@tok@kn}{\let\PY@bf=\textbf\def\PY@tc##1{\textcolor[rgb]{0.00,0.50,0.00}{##1}}}
\@namedef{PY@tok@kr}{\let\PY@bf=\textbf\def\PY@tc##1{\textcolor[rgb]{0.00,0.50,0.00}{##1}}}
\@namedef{PY@tok@bp}{\def\PY@tc##1{\textcolor[rgb]{0.00,0.50,0.00}{##1}}}
\@namedef{PY@tok@fm}{\def\PY@tc##1{\textcolor[rgb]{0.00,0.00,1.00}{##1}}}
\@namedef{PY@tok@vc}{\def\PY@tc##1{\textcolor[rgb]{0.10,0.09,0.49}{##1}}}
\@namedef{PY@tok@vg}{\def\PY@tc##1{\textcolor[rgb]{0.10,0.09,0.49}{##1}}}
\@namedef{PY@tok@vi}{\def\PY@tc##1{\textcolor[rgb]{0.10,0.09,0.49}{##1}}}
\@namedef{PY@tok@vm}{\def\PY@tc##1{\textcolor[rgb]{0.10,0.09,0.49}{##1}}}
\@namedef{PY@tok@sa}{\def\PY@tc##1{\textcolor[rgb]{0.73,0.13,0.13}{##1}}}
\@namedef{PY@tok@sb}{\def\PY@tc##1{\textcolor[rgb]{0.73,0.13,0.13}{##1}}}
\@namedef{PY@tok@sc}{\def\PY@tc##1{\textcolor[rgb]{0.73,0.13,0.13}{##1}}}
\@namedef{PY@tok@dl}{\def\PY@tc##1{\textcolor[rgb]{0.73,0.13,0.13}{##1}}}
\@namedef{PY@tok@s2}{\def\PY@tc##1{\textcolor[rgb]{0.73,0.13,0.13}{##1}}}
\@namedef{PY@tok@sh}{\def\PY@tc##1{\textcolor[rgb]{0.73,0.13,0.13}{##1}}}
\@namedef{PY@tok@s1}{\def\PY@tc##1{\textcolor[rgb]{0.73,0.13,0.13}{##1}}}
\@namedef{PY@tok@mb}{\def\PY@tc##1{\textcolor[rgb]{0.40,0.40,0.40}{##1}}}
\@namedef{PY@tok@mf}{\def\PY@tc##1{\textcolor[rgb]{0.40,0.40,0.40}{##1}}}
\@namedef{PY@tok@mh}{\def\PY@tc##1{\textcolor[rgb]{0.40,0.40,0.40}{##1}}}
\@namedef{PY@tok@mi}{\def\PY@tc##1{\textcolor[rgb]{0.40,0.40,0.40}{##1}}}
\@namedef{PY@tok@il}{\def\PY@tc##1{\textcolor[rgb]{0.40,0.40,0.40}{##1}}}
\@namedef{PY@tok@mo}{\def\PY@tc##1{\textcolor[rgb]{0.40,0.40,0.40}{##1}}}
\@namedef{PY@tok@ch}{\let\PY@it=\textit\def\PY@tc##1{\textcolor[rgb]{0.24,0.48,0.48}{##1}}}
\@namedef{PY@tok@cm}{\let\PY@it=\textit\def\PY@tc##1{\textcolor[rgb]{0.24,0.48,0.48}{##1}}}
\@namedef{PY@tok@cpf}{\let\PY@it=\textit\def\PY@tc##1{\textcolor[rgb]{0.24,0.48,0.48}{##1}}}
\@namedef{PY@tok@c1}{\let\PY@it=\textit\def\PY@tc##1{\textcolor[rgb]{0.24,0.48,0.48}{##1}}}
\@namedef{PY@tok@cs}{\let\PY@it=\textit\def\PY@tc##1{\textcolor[rgb]{0.24,0.48,0.48}{##1}}}

\def\PYZbs{\char`\\}
\def\PYZus{\char`\_}
\def\PYZob{\char`\{}
\def\PYZcb{\char`\}}
\def\PYZca{\char`\^}
\def\PYZam{\char`\&}
\def\PYZlt{\char`\<}
\def\PYZgt{\char`\>}
\def\PYZsh{\char`\#}
\def\PYZpc{\char`\%}
\def\PYZdl{\char`\$}
\def\PYZhy{\char`\-}
\def\PYZsq{\char`\'}
\def\PYZdq{\char`\"}
\def\PYZti{\char`\~}
% for compatibility with earlier versions
\def\PYZat{@}
\def\PYZlb{[}
\def\PYZrb{]}
\makeatother


    % For linebreaks inside Verbatim environment from package fancyvrb.
    \makeatletter
        \newbox\Wrappedcontinuationbox
        \newbox\Wrappedvisiblespacebox
        \newcommand*\Wrappedvisiblespace {\textcolor{red}{\textvisiblespace}}
        \newcommand*\Wrappedcontinuationsymbol {\textcolor{red}{\llap{\tiny$\m@th\hookrightarrow$}}}
        \newcommand*\Wrappedcontinuationindent {3ex }
        \newcommand*\Wrappedafterbreak {\kern\Wrappedcontinuationindent\copy\Wrappedcontinuationbox}
        % Take advantage of the already applied Pygments mark-up to insert
        % potential linebreaks for TeX processing.
        %        {, <, #, %, $, ' and ": go to next line.
        %        _, }, ^, &, >, - and ~: stay at end of broken line.
        % Use of \textquotesingle for straight quote.
        \newcommand*\Wrappedbreaksatspecials {%
            \def\PYGZus{\discretionary{\char`\_}{\Wrappedafterbreak}{\char`\_}}%
            \def\PYGZob{\discretionary{}{\Wrappedafterbreak\char`\{}{\char`\{}}%
            \def\PYGZcb{\discretionary{\char`\}}{\Wrappedafterbreak}{\char`\}}}%
            \def\PYGZca{\discretionary{\char`\^}{\Wrappedafterbreak}{\char`\^}}%
            \def\PYGZam{\discretionary{\char`\&}{\Wrappedafterbreak}{\char`\&}}%
            \def\PYGZlt{\discretionary{}{\Wrappedafterbreak\char`\<}{\char`\<}}%
            \def\PYGZgt{\discretionary{\char`\>}{\Wrappedafterbreak}{\char`\>}}%
            \def\PYGZsh{\discretionary{}{\Wrappedafterbreak\char`\#}{\char`\#}}%
            \def\PYGZpc{\discretionary{}{\Wrappedafterbreak\char`\%}{\char`\%}}%
            \def\PYGZdl{\discretionary{}{\Wrappedafterbreak\char`\$}{\char`\$}}%
            \def\PYGZhy{\discretionary{\char`\-}{\Wrappedafterbreak}{\char`\-}}%
            \def\PYGZsq{\discretionary{}{\Wrappedafterbreak\textquotesingle}{\textquotesingle}}%
            \def\PYGZdq{\discretionary{}{\Wrappedafterbreak\char`\"}{\char`\"}}%
            \def\PYGZti{\discretionary{\char`\~}{\Wrappedafterbreak}{\char`\~}}%
        }
        % Some characters . , ; ? ! / are not pygmentized.
        % This macro makes them "active" and they will insert potential linebreaks
        \newcommand*\Wrappedbreaksatpunct {%
            \lccode`\~`\.\lowercase{\def~}{\discretionary{\hbox{\char`\.}}{\Wrappedafterbreak}{\hbox{\char`\.}}}%
            \lccode`\~`\,\lowercase{\def~}{\discretionary{\hbox{\char`\,}}{\Wrappedafterbreak}{\hbox{\char`\,}}}%
            \lccode`\~`\;\lowercase{\def~}{\discretionary{\hbox{\char`\;}}{\Wrappedafterbreak}{\hbox{\char`\;}}}%
            \lccode`\~`\:\lowercase{\def~}{\discretionary{\hbox{\char`\:}}{\Wrappedafterbreak}{\hbox{\char`\:}}}%
            \lccode`\~`\?\lowercase{\def~}{\discretionary{\hbox{\char`\?}}{\Wrappedafterbreak}{\hbox{\char`\?}}}%
            \lccode`\~`\!\lowercase{\def~}{\discretionary{\hbox{\char`\!}}{\Wrappedafterbreak}{\hbox{\char`\!}}}%
            \lccode`\~`\/\lowercase{\def~}{\discretionary{\hbox{\char`\/}}{\Wrappedafterbreak}{\hbox{\char`\/}}}%
            \catcode`\.\active
            \catcode`\,\active
            \catcode`\;\active
            \catcode`\:\active
            \catcode`\?\active
            \catcode`\!\active
            \catcode`\/\active
            \lccode`\~`\~
        }
    \makeatother

    \let\OriginalVerbatim=\Verbatim
    \makeatletter
    \renewcommand{\Verbatim}[1][1]{%
        %\parskip\z@skip
        \sbox\Wrappedcontinuationbox {\Wrappedcontinuationsymbol}%
        \sbox\Wrappedvisiblespacebox {\FV@SetupFont\Wrappedvisiblespace}%
        \def\FancyVerbFormatLine ##1{\hsize\linewidth
            \vtop{\raggedright\hyphenpenalty\z@\exhyphenpenalty\z@
                \doublehyphendemerits\z@\finalhyphendemerits\z@
                \strut ##1\strut}%
        }%
        % If the linebreak is at a space, the latter will be displayed as visible
        % space at end of first line, and a continuation symbol starts next line.
        % Stretch/shrink are however usually zero for typewriter font.
        \def\FV@Space {%
            \nobreak\hskip\z@ plus\fontdimen3\font minus\fontdimen4\font
            \discretionary{\copy\Wrappedvisiblespacebox}{\Wrappedafterbreak}
            {\kern\fontdimen2\font}%
        }%

        % Allow breaks at special characters using \PYG... macros.
        \Wrappedbreaksatspecials
        % Breaks at punctuation characters . , ; ? ! and / need catcode=\active
        \OriginalVerbatim[#1,codes*=\Wrappedbreaksatpunct]%
    }
    \makeatother

    % Exact colors from NB
    \definecolor{incolor}{HTML}{303F9F}
    \definecolor{outcolor}{HTML}{D84315}
    \definecolor{cellborder}{HTML}{CFCFCF}
    \definecolor{cellbackground}{HTML}{F7F7F7}

    % prompt
    \makeatletter
    \newcommand{\boxspacing}{\kern\kvtcb@left@rule\kern\kvtcb@boxsep}
    \makeatother
    \newcommand{\prompt}[4]{
        {\ttfamily\llap{{\color{#2}[#3]:\hspace{3pt}#4}}\vspace{-\baselineskip}}
    }
    

    
    % Prevent overflowing lines due to hard-to-break entities
    \sloppy
    % Setup hyperref package
    \hypersetup{
      breaklinks=true,  % so long urls are correctly broken across lines
      colorlinks=true,
      urlcolor=urlcolor,
      linkcolor=linkcolor,
      citecolor=citecolor,
      }
    % Slightly bigger margins than the latex defaults
    
    \geometry{verbose,tmargin=1in,bmargin=1in,lmargin=1in,rmargin=1in}
    
    

\begin{document}
    
    \maketitle
    
    

    
    \section{Part 1: Quick Examples of
Python}\label{part-1-quick-examples-of-python}

Here are some quick examples of calculations and data analysis in
python. Examine the code. Make changes and see what they do.

\subsection{Python Notebooks and
Documentation}\label{python-notebooks-and-documentation}

Observe how every example can be documented with text and typeset math
equations. But, most important of all, all the math in your data
analysis is documented in the python code itself. Whatever you write in
the text, the code is the truth. If you made an error, you will find it
in the code. If you want to find out how a result was obtained from a
data set, the method will be in the code.

If you print this notebook out as a PDF file and include it in your
thesis, all the code will be visible for inspection by those who follow.

    \subsection{Hookes Law}\label{hookes-law}

Calculating the frequency of a C--H bond using Hooke's law.
\[\bar{\nu} = \frac{1}{2\pi c}\sqrt\frac{k}{\mu}\] where \(\mu\) is the
\emph{reduced mass} of the two atoms with masses \(m_1\) and \(m_2\)
\[\mu = \frac{m_1m_2}{m_1+m_2}\] and \(k\) is the force constant for the
vibration.

    \begin{tcolorbox}[breakable, size=fbox, boxrule=1pt, pad at break*=1mm,colback=cellbackground, colframe=cellborder]
\prompt{In}{incolor}{1}{\boxspacing}
\begin{Verbatim}[commandchars=\\\{\}]
\PY{c+c1}{\PYZsh{} Find frequency of IR vibration}
\PY{k+kn}{import} \PY{n+nn}{scipy}\PY{n+nn}{.}\PY{n+nn}{constants} \PY{k}{as} \PY{n+nn}{spc}   \PY{c+c1}{\PYZsh{} predefined constant library}
\PY{k+kn}{import} \PY{n+nn}{numpy} \PY{k}{as} \PY{n+nn}{np}              \PY{c+c1}{\PYZsh{} math tools}

\PY{n}{c} \PY{o}{=} \PY{n}{spc}\PY{o}{.}\PY{n}{c} \PY{o}{*} \PY{l+m+mi}{100}          \PY{c+c1}{\PYZsh{} speed of light: convert m.s\PYZca{}\PYZhy{}1 to cm.s\PYZca{}\PYZhy{}1}
\PY{n}{pi} \PY{o}{=} \PY{n}{spc}\PY{o}{.}\PY{n}{pi}              \PY{c+c1}{\PYZsh{} pi to many decimal places}
\PY{n}{N} \PY{o}{=} \PY{n}{spc}\PY{o}{.}\PY{n}{Avogadro}         \PY{c+c1}{\PYZsh{} Avogadro\PYZsq{}s number (mole\PYZca{}\PYZhy{}1)}

\PY{n}{k} \PY{o}{=} \PY{l+m+mf}{5E5}      \PY{c+c1}{\PYZsh{} 5x10\PYZca{}5 dynes/cm \PYZhy{}\PYZhy{} the force constant of a C\PYZhy{}H bond stretch}
\PY{n}{m1} \PY{o}{=} \PY{l+m+mi}{1} \PY{o}{/} \PY{n}{N}   \PY{c+c1}{\PYZsh{} mass of an individual carbon atom}
\PY{n}{m2} \PY{o}{=} \PY{l+m+mi}{12} \PY{o}{/} \PY{n}{N}  \PY{c+c1}{\PYZsh{} mass of an individual hydrogen atom}

\PY{n}{u} \PY{o}{=} \PY{n}{m1}\PY{o}{*}\PY{n}{m2} \PY{o}{/} \PY{p}{(}\PY{n}{m1}\PY{o}{+}\PY{n}{m2}\PY{p}{)}                  \PY{c+c1}{\PYZsh{} Reduced mass}
\PY{n}{freq} \PY{o}{=} \PY{l+m+mi}{1} \PY{o}{/} \PY{p}{(}\PY{l+m+mi}{2}\PY{o}{*}\PY{n}{pi}\PY{o}{*}\PY{n}{c}\PY{p}{)} \PY{o}{*} \PY{n}{np}\PY{o}{.}\PY{n}{sqrt}\PY{p}{(}\PY{n}{k}\PY{o}{/}\PY{n}{u}\PY{p}{)}   \PY{c+c1}{\PYZsh{} Hooke\PYZsq{}s law}

\PY{n+nb}{print}\PY{p}{(}\PY{l+s+sa}{f}\PY{l+s+s2}{\PYZdq{}}\PY{l+s+s2}{The frequency is }\PY{l+s+si}{\PYZob{}}\PY{n}{freq}\PY{l+s+si}{:}\PY{l+s+s2}{0.1f}\PY{l+s+si}{\PYZcb{}}\PY{l+s+s2}{ cm\PYZca{}\PYZhy{}1}\PY{l+s+s2}{\PYZdq{}}\PY{p}{)}
\end{Verbatim}
\end{tcolorbox}

    \begin{Verbatim}[commandchars=\\\{\}]
The frequency is 3032.1 cm\^{}-1
    \end{Verbatim}

    \subsection{Plotting Data}\label{plotting-data}

The code below takes two lists of numbers, \(x\) and \(y\), and plots
them on a graph. The \texttt{plt.plot()} command makes the plot, the
rest are all style. Examine the code. Change things and break things.
Have fun.

The data is for the enzyme-catalyzed hydrolysis of p-nitrophenylacetate
at pH 7.1 in the presence of an extract of pineapple juice. The enzyme
\emph{pectinmethylesterase}, EC 3.1.1.11, is most likely the source of
the catalysis.

Below is a simple plot. Observe all we need is \(x\) and \(y\) data and
a plot command. Its that simple.

    \begin{tcolorbox}[breakable, size=fbox, boxrule=1pt, pad at break*=1mm,colback=cellbackground, colframe=cellborder]
\prompt{In}{incolor}{3}{\boxspacing}
\begin{Verbatim}[commandchars=\\\{\}]
\PY{k+kn}{import} \PY{n+nn}{matplotlib}\PY{n+nn}{.}\PY{n+nn}{pyplot} \PY{k}{as} \PY{n+nn}{plt}    \PY{c+c1}{\PYZsh{} plotting tools}
\PY{n}{conc} \PY{o}{=} \PY{p}{[}\PY{l+m+mi}{1}\PY{p}{,}  \PY{l+m+mi}{2}\PY{p}{,}  \PY{l+m+mi}{3}\PY{p}{,}  \PY{l+m+mi}{5}\PY{p}{,}  \PY{l+m+mi}{7}\PY{p}{,}  \PY{l+m+mi}{9}\PY{p}{,} \PY{l+m+mi}{10}\PY{p}{,} \PY{l+m+mi}{15}\PY{p}{]}                  \PY{c+c1}{\PYZsh{} units are mM}
\PY{n}{rate} \PY{o}{=} \PY{p}{[}\PY{l+m+mf}{7.3}\PY{p}{,} \PY{l+m+mf}{20.7}\PY{p}{,} \PY{l+m+mf}{24.1}\PY{p}{,} \PY{l+m+mf}{34.3}\PY{p}{,} \PY{l+m+mf}{39.6}\PY{p}{,} \PY{l+m+mf}{48.2}\PY{p}{,} \PY{l+m+mf}{47.0}\PY{p}{,} \PY{l+m+mf}{55.2}\PY{p}{]}  \PY{c+c1}{\PYZsh{} units are uM/min}
\PY{n}{plt}\PY{o}{.}\PY{n}{plot}\PY{p}{(}\PY{n}{conc}\PY{p}{,} \PY{n}{rate}\PY{p}{,} \PY{l+s+s2}{\PYZdq{}}\PY{l+s+s2}{ko}\PY{l+s+s2}{\PYZdq{}}\PY{p}{)}          \PY{c+c1}{\PYZsh{} plot the data as black \PYZdq{}k\PYZdq{} points \PYZdq{}o\PYZdq{}}
\PY{n}{plt}\PY{o}{.}\PY{n}{show}\PY{p}{(}\PY{p}{)}
\end{Verbatim}
\end{tcolorbox}

    \begin{center}
    \adjustimage{max size={0.9\linewidth}{0.9\paperheight}}{01X_PlotBasics_files/01X_PlotBasics_4_0.png}
    \end{center}
    { \hspace*{\fill} \\}
    
    \subsubsection{The Sample Plot with
Style}\label{the-sample-plot-with-style}

    \begin{tcolorbox}[breakable, size=fbox, boxrule=1pt, pad at break*=1mm,colback=cellbackground, colframe=cellborder]
\prompt{In}{incolor}{4}{\boxspacing}
\begin{Verbatim}[commandchars=\\\{\}]
\PY{c+c1}{\PYZsh{}\PYZsh{}\PYZsh{} Setup environment for Google Colab}

\PY{k+kn}{import} \PY{n+nn}{matplotlib}\PY{n+nn}{.}\PY{n+nn}{pyplot} \PY{k}{as} \PY{n+nn}{plt}    \PY{c+c1}{\PYZsh{} plotting tools}
\PY{o}{!}mkdir\PY{+w}{ }plots

\PY{c+c1}{\PYZsh{}\PYZsh{}\PYZsh{} This is the data}
\PY{n}{conc} \PY{o}{=} \PY{p}{[}\PY{l+m+mi}{1}\PY{p}{,}  \PY{l+m+mi}{2}\PY{p}{,}  \PY{l+m+mi}{3}\PY{p}{,}  \PY{l+m+mi}{5}\PY{p}{,}  \PY{l+m+mi}{7}\PY{p}{,}  \PY{l+m+mi}{9}\PY{p}{,} \PY{l+m+mi}{10}\PY{p}{,} \PY{l+m+mi}{15}\PY{p}{]}                  \PY{c+c1}{\PYZsh{} units are mM}
\PY{n}{rate} \PY{o}{=} \PY{p}{[}\PY{l+m+mf}{7.3}\PY{p}{,} \PY{l+m+mf}{20.7}\PY{p}{,} \PY{l+m+mf}{24.1}\PY{p}{,} \PY{l+m+mf}{34.3}\PY{p}{,} \PY{l+m+mf}{39.6}\PY{p}{,} \PY{l+m+mf}{48.2}\PY{p}{,} \PY{l+m+mf}{47.0}\PY{p}{,} \PY{l+m+mf}{55.2}\PY{p}{]}  \PY{c+c1}{\PYZsh{} units are uM/min}

\PY{c+c1}{\PYZsh{}\PYZsh{}\PYZsh{}\PYZsh{}\PYZsh{} PLOT COMMANDS \PYZsh{}\PYZsh{}\PYZsh{}\PYZsh{}\PYZsh{}}

\PY{n}{plt}\PY{o}{.}\PY{n}{figure}\PY{p}{(}\PY{n}{figsize}\PY{o}{=}\PY{p}{(}\PY{l+m+mi}{4}\PY{p}{,}\PY{l+m+mi}{4}\PY{p}{)}\PY{p}{)}  \PY{c+c1}{\PYZsh{} create a new blank plot}

\PY{n}{plt}\PY{o}{.}\PY{n}{plot}\PY{p}{(}\PY{n}{conc}\PY{p}{,} \PY{n}{rate}\PY{p}{,} \PY{l+s+s2}{\PYZdq{}}\PY{l+s+s2}{ko}\PY{l+s+s2}{\PYZdq{}}\PY{p}{)} \PY{c+c1}{\PYZsh{} plot the data as black \PYZdq{}k\PYZdq{} points \PYZdq{}o\PYZdq{}}

\PY{n}{plt}\PY{o}{.}\PY{n}{xlim}\PY{p}{(}\PY{l+m+mi}{0}\PY{p}{,}\PY{k+kc}{None}\PY{p}{)}           \PY{c+c1}{\PYZsh{} set axis limits so the plot starts at the origin}
\PY{n}{plt}\PY{o}{.}\PY{n}{ylim}\PY{p}{(}\PY{l+m+mi}{0}\PY{p}{,}\PY{k+kc}{None}\PY{p}{)}
\PY{n}{plt}\PY{o}{.}\PY{n}{xlabel}\PY{p}{(}\PY{l+s+sa}{r}\PY{l+s+s2}{\PYZdq{}}\PY{l+s+s2}{\PYZdl{}[S]\PYZdl{}}\PY{l+s+s2}{\PYZdq{}}\PY{p}{)}       \PY{c+c1}{\PYZsh{} label the two axis}
\PY{n}{plt}\PY{o}{.}\PY{n}{ylabel}\PY{p}{(}\PY{l+s+sa}{r}\PY{l+s+s2}{\PYZdq{}}\PY{l+s+s2}{\PYZdl{}}\PY{l+s+s2}{\PYZbs{}}\PY{l+s+s2}{nu\PYZdl{}}\PY{l+s+s2}{\PYZdq{}}\PY{p}{)}
\PY{n}{plt}\PY{o}{.}\PY{n}{title}\PY{p}{(}\PY{l+s+s2}{\PYZdq{}}\PY{l+s+s2}{Michaelis\PYZhy{}Menten plot}\PY{l+s+s2}{\PYZdq{}}\PY{p}{)}

\PY{n}{plt}\PY{o}{.}\PY{n}{tight\PYZus{}layout}\PY{p}{(}\PY{p}{)}         \PY{c+c1}{\PYZsh{} Prevents axis lables from falling off edge of plot}
\PY{n}{plt}\PY{o}{.}\PY{n}{savefig}\PY{p}{(}\PY{l+s+s2}{\PYZdq{}}\PY{l+s+s2}{plots/basics\PYZus{}plot1.pdf}\PY{l+s+s2}{\PYZdq{}}\PY{p}{)}   \PY{c+c1}{\PYZsh{} save the plot to this file}
\PY{n}{plt}\PY{o}{.}\PY{n}{show}\PY{p}{(}\PY{p}{)}                 \PY{c+c1}{\PYZsh{} show the plot in this notebook and clear it}
\end{Verbatim}
\end{tcolorbox}

    \begin{Verbatim}[commandchars=\\\{\}]
mkdir: plots: File exists
    \end{Verbatim}

    \begin{center}
    \adjustimage{max size={0.9\linewidth}{0.9\paperheight}}{01X_PlotBasics_files/01X_PlotBasics_6_1.png}
    \end{center}
    { \hspace*{\fill} \\}
    
    \subsection{Data Analysis - Linear
Regression}\label{data-analysis---linear-regression}

A famous (or infamous) method for analyzing enzyme kinetics is the
Lineweaver-Burke equation. It is a double-reciprocal plot. We must
convert the \(x\) and \(y\) values to \(1/x\) and \(1/y\). The
\(y\)-intercept will be \(1/V_{max}\) and the slope will be
\(K_M/V_{max}\). From this plot we can determine these two kinetic
parameters for the enzyme and its substrate.

The code below demonstrates these calculations, the plot and a linear
regression line fit to obtain the slope and intercept.

    \begin{tcolorbox}[breakable, size=fbox, boxrule=1pt, pad at break*=1mm,colback=cellbackground, colframe=cellborder]
\prompt{In}{incolor}{5}{\boxspacing}
\begin{Verbatim}[commandchars=\\\{\}]
\PY{k+kn}{import} \PY{n+nn}{matplotlib}\PY{n+nn}{.}\PY{n+nn}{pyplot} \PY{k}{as} \PY{n+nn}{plt}    \PY{c+c1}{\PYZsh{} plotting tools}
\PY{k+kn}{import} \PY{n+nn}{numpy} \PY{k}{as} \PY{n+nn}{np}                 \PY{c+c1}{\PYZsh{} we need the array object}
\PY{k+kn}{import} \PY{n+nn}{scipy}

\PY{n}{conc} \PY{o}{=} \PY{p}{[}\PY{l+m+mi}{1}\PY{p}{,}  \PY{l+m+mi}{2}\PY{p}{,}  \PY{l+m+mi}{3}\PY{p}{,}  \PY{l+m+mi}{5}\PY{p}{,}  \PY{l+m+mi}{7}\PY{p}{,}  \PY{l+m+mi}{9}\PY{p}{,} \PY{l+m+mi}{10}\PY{p}{,} \PY{l+m+mi}{15}\PY{p}{]}                  \PY{c+c1}{\PYZsh{} units are mM}
\PY{n}{rate} \PY{o}{=} \PY{p}{[}\PY{l+m+mf}{7.3}\PY{p}{,} \PY{l+m+mf}{20.7}\PY{p}{,} \PY{l+m+mf}{24.1}\PY{p}{,} \PY{l+m+mf}{34.3}\PY{p}{,} \PY{l+m+mf}{39.6}\PY{p}{,} \PY{l+m+mf}{48.2}\PY{p}{,} \PY{l+m+mf}{47.0}\PY{p}{,} \PY{l+m+mf}{55.2}\PY{p}{]}  \PY{c+c1}{\PYZsh{} units are uM/min}

\PY{c+c1}{\PYZsh{}\PYZsh{}\PYZsh{}\PYZsh{}\PYZsh{} MATH COMMANDS \PYZsh{}\PYZsh{}\PYZsh{}\PYZsh{}\PYZsh{}}

\PY{n}{conc} \PY{o}{=} \PY{n}{np}\PY{o}{.}\PY{n}{array}\PY{p}{(}\PY{n}{conc}\PY{p}{)}  \PY{c+c1}{\PYZsh{} convert lists to numpy arrays to allow math operations }
\PY{n}{rate} \PY{o}{=} \PY{n}{np}\PY{o}{.}\PY{n}{array}\PY{p}{(}\PY{n}{rate}\PY{p}{)}  \PY{c+c1}{\PYZsh{}  that lists do not possess.}

\PY{n}{x} \PY{o}{=} \PY{l+m+mi}{1}\PY{o}{/}\PY{n}{conc}   \PY{c+c1}{\PYZsh{} reciprocal values for double reciprocal plot}
\PY{n}{y} \PY{o}{=} \PY{l+m+mi}{1}\PY{o}{/}\PY{n}{rate}

\PY{c+c1}{\PYZsh{} calculate line fit and returns an object containing parameters }
\PY{n}{result} \PY{o}{=} \PY{n}{scipy}\PY{o}{.}\PY{n}{stats}\PY{o}{.}\PY{n}{linregress}\PY{p}{(}\PY{n}{x}\PY{p}{,}\PY{n}{y}\PY{p}{)}  

\PY{n}{intercept} \PY{o}{=} \PY{n}{result}\PY{o}{.}\PY{n}{intercept}   \PY{c+c1}{\PYZsh{} get the slope and intercept from result object}
\PY{n}{slope} \PY{o}{=} \PY{n}{result}\PY{o}{.}\PY{n}{slope}

\PY{n}{y\PYZus{}predicted} \PY{o}{=} \PY{n}{slope} \PY{o}{*} \PY{n}{x} \PY{o}{+} \PY{n}{intercept}   \PY{c+c1}{\PYZsh{} calculate the predicted line }

\PY{c+c1}{\PYZsh{}\PYZsh{}\PYZsh{}\PYZsh{}\PYZsh{} PLOT COMMANDS \PYZsh{}\PYZsh{}\PYZsh{}\PYZsh{}\PYZsh{}}

\PY{n}{plt}\PY{o}{.}\PY{n}{figure}\PY{p}{(}\PY{n}{figsize}\PY{o}{=}\PY{p}{(}\PY{l+m+mi}{4}\PY{p}{,}\PY{l+m+mi}{4}\PY{p}{)}\PY{p}{)}       \PY{c+c1}{\PYZsh{} create a new blank plot}

\PY{n}{plt}\PY{o}{.}\PY{n}{plot}\PY{p}{(}\PY{n}{x}\PY{p}{,} \PY{n}{y}\PY{p}{,} \PY{l+s+s2}{\PYZdq{}}\PY{l+s+s2}{ko}\PY{l+s+s2}{\PYZdq{}}\PY{p}{)}      \PY{c+c1}{\PYZsh{} plot the data as black \PYZdq{}k\PYZdq{} points \PYZdq{}o\PYZdq{}}
\PY{n}{plt}\PY{o}{.}\PY{n}{plot}\PY{p}{(}\PY{n}{x}\PY{p}{,} \PY{n}{y\PYZus{}predicted}\PY{p}{,} \PY{l+s+s2}{\PYZdq{}}\PY{l+s+s2}{k\PYZhy{}}\PY{l+s+s2}{\PYZdq{}}\PY{p}{)}  \PY{c+c1}{\PYZsh{} plot the predicted line as black \PYZdq{}k\PYZdq{} line \PYZdq{}\PYZhy{}\PYZdq{}}

\PY{n}{plt}\PY{o}{.}\PY{n}{xlim}\PY{p}{(}\PY{l+m+mi}{0}\PY{p}{,}\PY{k+kc}{None}\PY{p}{)}          \PY{c+c1}{\PYZsh{} set axis limits so the plot starts at the origin}
\PY{n}{plt}\PY{o}{.}\PY{n}{ylim}\PY{p}{(}\PY{l+m+mi}{0}\PY{p}{,}\PY{l+m+mf}{0.15}\PY{p}{)}
\PY{n}{plt}\PY{o}{.}\PY{n}{xlabel}\PY{p}{(}\PY{l+s+sa}{r}\PY{l+s+s2}{\PYZdq{}}\PY{l+s+s2}{1/[S]}\PY{l+s+s2}{\PYZdq{}}\PY{p}{)}      \PY{c+c1}{\PYZsh{} label the two axis}
\PY{n}{plt}\PY{o}{.}\PY{n}{ylabel}\PY{p}{(}\PY{l+s+sa}{r}\PY{l+s+s2}{\PYZdq{}}\PY{l+s+s2}{\PYZdl{}1/}\PY{l+s+s2}{\PYZbs{}}\PY{l+s+s2}{nu\PYZdl{}}\PY{l+s+s2}{\PYZdq{}}\PY{p}{)}
\PY{n}{plt}\PY{o}{.}\PY{n}{title}\PY{p}{(}\PY{l+s+sa}{r}\PY{l+s+s2}{\PYZdq{}}\PY{l+s+s2}{Lineweaver\PYZhy{}Burke plot}\PY{l+s+s2}{\PYZdq{}}\PY{p}{)}

\PY{n}{plt}\PY{o}{.}\PY{n}{tight\PYZus{}layout}\PY{p}{(}\PY{p}{)}        \PY{c+c1}{\PYZsh{} Prevents axis lables from falling off edge of plot}
\PY{n}{plt}\PY{o}{.}\PY{n}{savefig}\PY{p}{(}\PY{l+s+s2}{\PYZdq{}}\PY{l+s+s2}{plots/basics\PYZus{}plot2.pdf}\PY{l+s+s2}{\PYZdq{}}\PY{p}{)}   \PY{c+c1}{\PYZsh{} save the plot to this file}
\PY{n}{plt}\PY{o}{.}\PY{n}{show}\PY{p}{(}\PY{p}{)}                \PY{c+c1}{\PYZsh{} show the plot in this notebook and clear it}

\PY{n}{vmax\PYZus{}lb} \PY{o}{=} \PY{l+m+mi}{1}\PY{o}{/}\PY{n}{intercept}
\PY{n}{KM\PYZus{}lb} \PY{o}{=} \PY{n}{slope} \PY{o}{*} \PY{n}{vmax\PYZus{}lb}

\PY{c+c1}{\PYZsh{}\PYZsh{}\PYZsh{}\PYZsh{}\PYZsh{} PRINT REPORT \PYZsh{}\PYZsh{}\PYZsh{}\PYZsh{}\PYZsh{}}

\PY{n+nb}{print}\PY{p}{(}\PY{l+s+sa}{f}\PY{l+s+s2}{\PYZdq{}}\PY{l+s+s2}{The intercept is }\PY{l+s+si}{\PYZob{}}\PY{n}{intercept}\PY{l+s+si}{:}\PY{l+s+s2}{0.4f}\PY{l+s+si}{\PYZcb{}}\PY{l+s+s2}{\PYZdq{}}\PY{p}{)} 
\PY{n+nb}{print}\PY{p}{(}\PY{l+s+sa}{f}\PY{l+s+s2}{\PYZdq{}}\PY{l+s+s2}{The slope is }\PY{l+s+si}{\PYZob{}}\PY{n}{slope}\PY{l+s+si}{:}\PY{l+s+s2}{0.4f}\PY{l+s+si}{\PYZcb{}}\PY{l+s+s2}{\PYZdq{}}\PY{p}{)} 
\PY{n+nb}{print}\PY{p}{(}\PY{l+s+sa}{f}\PY{l+s+s2}{\PYZdq{}}\PY{l+s+s2}{The RSQ is }\PY{l+s+si}{\PYZob{}}\PY{n}{result}\PY{o}{.}\PY{n}{rvalue}\PY{+w}{ }\PY{o}{*}\PY{o}{*}\PY{+w}{ }\PY{l+m+mi}{2}\PY{l+s+si}{:}\PY{l+s+s2}{0.2f}\PY{l+s+si}{\PYZcb{}}\PY{l+s+s2}{\PYZdq{}}\PY{p}{)}
\PY{n+nb}{print}\PY{p}{(}\PY{l+s+sa}{f}\PY{l+s+s2}{\PYZdq{}}\PY{l+s+s2}{The Vmax is }\PY{l+s+si}{\PYZob{}}\PY{n}{vmax\PYZus{}lb}\PY{l+s+si}{:}\PY{l+s+s2}{0.4f}\PY{l+s+si}{\PYZcb{}}\PY{l+s+s2}{\PYZdq{}}\PY{p}{)} 
\PY{n+nb}{print}\PY{p}{(}\PY{l+s+sa}{f}\PY{l+s+s2}{\PYZdq{}}\PY{l+s+s2}{The KM is }\PY{l+s+si}{\PYZob{}}\PY{n}{KM\PYZus{}lb}\PY{l+s+si}{:}\PY{l+s+s2}{0.4f}\PY{l+s+si}{\PYZcb{}}\PY{l+s+s2}{\PYZdq{}}\PY{p}{)} 
\end{Verbatim}
\end{tcolorbox}

    \begin{center}
    \adjustimage{max size={0.9\linewidth}{0.9\paperheight}}{01X_PlotBasics_files/01X_PlotBasics_8_0.png}
    \end{center}
    { \hspace*{\fill} \\}
    
    \begin{Verbatim}[commandchars=\\\{\}]
The intercept is 0.0050
The slope is 0.1227
The RSQ is 0.96
The Vmax is 198.8730
The KM is 24.4023
    \end{Verbatim}

    \subsection{Data Analysis - Non-linear
Regression}\label{data-analysis---non-linear-regression}

We can fit data to any function and so are not limited to linear fits.
The \texttt{scipy.optimize.curve\_fit} tool will optimized parameters to
obtain the best fit of x and y to a function that you define.

The code below demonstrates these calculations, the plot and a curve fit
to obtain the best-fit parameters.

    \begin{tcolorbox}[breakable, size=fbox, boxrule=1pt, pad at break*=1mm,colback=cellbackground, colframe=cellborder]
\prompt{In}{incolor}{6}{\boxspacing}
\begin{Verbatim}[commandchars=\\\{\}]
\PY{k+kn}{import} \PY{n+nn}{matplotlib}\PY{n+nn}{.}\PY{n+nn}{pyplot} \PY{k}{as} \PY{n+nn}{plt}    \PY{c+c1}{\PYZsh{} plotting tools}
\PY{k+kn}{import} \PY{n+nn}{numpy} \PY{k}{as} \PY{n+nn}{np}                 \PY{c+c1}{\PYZsh{} we need the array object}
\PY{k+kn}{import} \PY{n+nn}{scipy}                       \PY{c+c1}{\PYZsh{} tools for science}

\PY{k}{def} \PY{n+nf}{MM}\PY{p}{(}\PY{n}{S}\PY{p}{,} \PY{n}{KM}\PY{p}{,} \PY{n}{Vmax}\PY{p}{)}\PY{p}{:}
    \PY{n}{rate} \PY{o}{=} \PY{n}{Vmax} \PY{o}{*} \PY{n}{S}\PY{o}{/}\PY{p}{(}\PY{n}{KM}\PY{o}{+}\PY{n}{S}\PY{p}{)}
    \PY{k}{return}\PY{p}{(}\PY{n}{rate}\PY{p}{)}

\PY{k}{def} \PY{n+nf}{MMplot}\PY{p}{(}\PY{n}{S}\PY{p}{,} \PY{n}{Vmax}\PY{p}{,} \PY{n}{KM}\PY{p}{)}\PY{p}{:}
    \PY{n}{v} \PY{o}{=} \PY{n}{Vmax} \PY{o}{*} \PY{n}{S} \PY{o}{/} \PY{p}{(}\PY{n}{S} \PY{o}{+} \PY{n}{KM}\PY{p}{)}
    \PY{k}{return}\PY{p}{(}\PY{n}{v}\PY{p}{)}

\PY{n}{conc} \PY{o}{=} \PY{p}{[}\PY{l+m+mf}{1.}\PY{p}{,}  \PY{l+m+mf}{2.}\PY{p}{,}  \PY{l+m+mf}{3.}\PY{p}{,}  \PY{l+m+mf}{5.}\PY{p}{,}  \PY{l+m+mf}{7.}\PY{p}{,}  \PY{l+m+mf}{9.}\PY{p}{,} \PY{l+m+mf}{10.}\PY{p}{,} \PY{l+m+mf}{15.}\PY{p}{]}          \PY{c+c1}{\PYZsh{} units are mM}
\PY{n}{rate} \PY{o}{=} \PY{p}{[}\PY{l+m+mf}{7.3}\PY{p}{,} \PY{l+m+mf}{20.7}\PY{p}{,} \PY{l+m+mf}{24.1}\PY{p}{,} \PY{l+m+mf}{34.3}\PY{p}{,} \PY{l+m+mf}{39.6}\PY{p}{,} \PY{l+m+mf}{48.2}\PY{p}{,} \PY{l+m+mf}{47.0}\PY{p}{,} \PY{l+m+mf}{55.2}\PY{p}{]}  \PY{c+c1}{\PYZsh{} units are uM/min}

\PY{c+c1}{\PYZsh{}\PYZsh{}\PYZsh{}\PYZsh{}\PYZsh{} MATH COMMANDS \PYZsh{}\PYZsh{}\PYZsh{}\PYZsh{}\PYZsh{}}

\PY{n}{conc} \PY{o}{=} \PY{n}{np}\PY{o}{.}\PY{n}{array}\PY{p}{(}\PY{n}{conc}\PY{p}{)}  \PY{c+c1}{\PYZsh{} convert list to numpy arrays to enable math operations }
\PY{n}{rate} \PY{o}{=} \PY{n}{np}\PY{o}{.}\PY{n}{array}\PY{p}{(}\PY{n}{rate}\PY{p}{)}  \PY{c+c1}{\PYZsh{}  that lists do not possess.}

\PY{c+c1}{\PYZsh{} calculate line fit and returns a list containing parameters}
\PY{n}{result} \PY{o}{=} \PY{n}{scipy}\PY{o}{.}\PY{n}{optimize}\PY{o}{.}\PY{n}{curve\PYZus{}fit}\PY{p}{(}\PY{n}{MM}\PY{p}{,} \PY{n}{conc}\PY{p}{,} \PY{n}{rate}\PY{p}{,} \PY{n}{p0} \PY{o}{=} \PY{p}{[}\PY{l+m+mi}{5}\PY{p}{,}\PY{l+m+mi}{70}\PY{p}{]}\PY{p}{)}   

\PY{c+c1}{\PYZsh{} extract the optimized parameter list (popt) and covariance matrix (pcov)}
\PY{p}{[}\PY{n}{popt}\PY{p}{,} \PY{n}{pcov}\PY{p}{]} \PY{o}{=} \PY{n}{result}  
\PY{p}{[}\PY{n}{KM}\PY{p}{,} \PY{n}{Vmax}\PY{p}{]} \PY{o}{=} \PY{n}{popt}      \PY{c+c1}{\PYZsh{} KM and Vmax were the two mitems i n the popt list}

\PY{n}{x\PYZus{}fit} \PY{o}{=} \PY{n}{np}\PY{o}{.}\PY{n}{linspace}\PY{p}{(}\PY{l+m+mi}{0}\PY{p}{,} \PY{n}{np}\PY{o}{.}\PY{n}{max}\PY{p}{(}\PY{n}{conc}\PY{p}{)}\PY{p}{,} \PY{l+m+mi}{100}\PY{p}{)}
\PY{n}{y\PYZus{}fit} \PY{o}{=} \PY{n}{MM}\PY{p}{(}\PY{n}{x\PYZus{}fit}\PY{p}{,} \PY{n}{KM}\PY{p}{,} \PY{n}{Vmax}\PY{p}{)}

\PY{c+c1}{\PYZsh{}\PYZsh{}\PYZsh{}\PYZsh{}\PYZsh{} PLOT COMMANDS \PYZsh{}\PYZsh{}\PYZsh{}\PYZsh{}\PYZsh{}}
\PY{n}{plt}\PY{o}{.}\PY{n}{figure}\PY{p}{(}\PY{n}{figsize}\PY{o}{=}\PY{p}{(}\PY{l+m+mi}{4}\PY{p}{,}\PY{l+m+mi}{4}\PY{p}{)}\PY{p}{)}    \PY{c+c1}{\PYZsh{} create a new blank plot}

\PY{n}{plt}\PY{o}{.}\PY{n}{plot}\PY{p}{(}\PY{n}{x\PYZus{}fit}\PY{p}{,} \PY{n}{y\PYZus{}fit}\PY{p}{,} \PY{l+s+s2}{\PYZdq{}}\PY{l+s+s2}{k\PYZhy{}}\PY{l+s+s2}{\PYZdq{}}\PY{p}{)} \PY{c+c1}{\PYZsh{} plot the predicted line as black \PYZdq{}k\PYZdq{} line \PYZdq{}\PYZhy{}\PYZdq{}}

\PY{c+c1}{\PYZsh{} plot the LB line as red for comparison}
\PY{n}{plt}\PY{o}{.}\PY{n}{plot}\PY{p}{(}\PY{n}{x\PYZus{}fit}\PY{p}{,} \PY{n}{MM}\PY{p}{(}\PY{n}{x\PYZus{}fit}\PY{p}{,} \PY{n}{KM\PYZus{}lb}\PY{p}{,} \PY{n}{vmax\PYZus{}lb}\PY{p}{)}\PY{p}{,} \PY{l+s+s2}{\PYZdq{}}\PY{l+s+s2}{r\PYZhy{}}\PY{l+s+s2}{\PYZdq{}}\PY{p}{)}  

\PY{n}{plt}\PY{o}{.}\PY{n}{plot}\PY{p}{(}\PY{n}{conc}\PY{p}{,} \PY{n}{rate}\PY{p}{,} \PY{l+s+s2}{\PYZdq{}}\PY{l+s+s2}{ko}\PY{l+s+s2}{\PYZdq{}}\PY{p}{)} \PY{c+c1}{\PYZsh{} plot the data as black \PYZdq{}k\PYZdq{} points \PYZdq{}o\PYZdq{}}
\PY{n}{plt}\PY{o}{.}\PY{n}{xlim}\PY{p}{(}\PY{l+m+mi}{0}\PY{p}{,}\PY{k+kc}{None}\PY{p}{)}           \PY{c+c1}{\PYZsh{} set axis limits so the plot starts at the origin}
\PY{n}{plt}\PY{o}{.}\PY{n}{ylim}\PY{p}{(}\PY{l+m+mi}{0}\PY{p}{,}\PY{l+m+mi}{80}\PY{p}{)}
\PY{n}{plt}\PY{o}{.}\PY{n}{xlabel}\PY{p}{(}\PY{l+s+sa}{r}\PY{l+s+s2}{\PYZdq{}}\PY{l+s+s2}{[S]}\PY{l+s+s2}{\PYZdq{}}\PY{p}{)}         \PY{c+c1}{\PYZsh{} label the two axis}
\PY{n}{plt}\PY{o}{.}\PY{n}{ylabel}\PY{p}{(}\PY{l+s+sa}{r}\PY{l+s+s2}{\PYZdq{}}\PY{l+s+s2}{\PYZdl{}}\PY{l+s+s2}{\PYZbs{}}\PY{l+s+s2}{nu\PYZdl{}}\PY{l+s+s2}{\PYZdq{}}\PY{p}{)}
\PY{n}{plt}\PY{o}{.}\PY{n}{title}\PY{p}{(}\PY{l+s+sa}{r}\PY{l+s+s2}{\PYZdq{}}\PY{l+s+s2}{Michaelis\PYZhy{}Menten plot}\PY{l+s+s2}{\PYZdq{}}\PY{p}{)}

\PY{n}{plt}\PY{o}{.}\PY{n}{tight\PYZus{}layout}\PY{p}{(}\PY{p}{)}         \PY{c+c1}{\PYZsh{} Prevents axis lables from falling off edge of plot}
\PY{n}{plt}\PY{o}{.}\PY{n}{savefig}\PY{p}{(}\PY{l+s+s2}{\PYZdq{}}\PY{l+s+s2}{plots/basics\PYZus{}plot3.pdf}\PY{l+s+s2}{\PYZdq{}}\PY{p}{)}   \PY{c+c1}{\PYZsh{} save the plot to this file}
\PY{n}{plt}\PY{o}{.}\PY{n}{show}\PY{p}{(}\PY{p}{)}                 \PY{c+c1}{\PYZsh{} show the plot in this notebook and clear it}

\PY{c+c1}{\PYZsh{}\PYZsh{}\PYZsh{}\PYZsh{}\PYZsh{} PRINT REPORT \PYZsh{}\PYZsh{}\PYZsh{}\PYZsh{}\PYZsh{}}

\PY{n+nb}{print}\PY{p}{(}\PY{l+s+sa}{f}\PY{l+s+s2}{\PYZdq{}}\PY{l+s+s2}{The Vmax is }\PY{l+s+si}{\PYZob{}}\PY{n}{Vmax}\PY{l+s+si}{:}\PY{l+s+s2}{0.4f}\PY{l+s+si}{\PYZcb{}}\PY{l+s+s2}{\PYZdq{}}\PY{p}{)} 
\PY{n+nb}{print}\PY{p}{(}\PY{l+s+sa}{f}\PY{l+s+s2}{\PYZdq{}}\PY{l+s+s2}{The KM is }\PY{l+s+si}{\PYZob{}}\PY{n}{KM}\PY{l+s+si}{:}\PY{l+s+s2}{0.4f}\PY{l+s+si}{\PYZcb{}}\PY{l+s+s2}{\PYZdq{}}\PY{p}{)} 
\PY{n+nb}{print}\PY{p}{(}\PY{l+s+s2}{\PYZdq{}}\PY{l+s+s2}{The red line is the predicted curve using Lineweaver\PYZhy{}Burke results.}\PY{l+s+s2}{\PYZdq{}}\PY{p}{)}
\end{Verbatim}
\end{tcolorbox}

    \begin{center}
    \adjustimage{max size={0.9\linewidth}{0.9\paperheight}}{01X_PlotBasics_files/01X_PlotBasics_10_0.png}
    \end{center}
    { \hspace*{\fill} \\}
    
    \begin{Verbatim}[commandchars=\\\{\}]
The Vmax is 81.0018
The KM is 6.8834
The red line is the predicted curve using Lineweaver-Burke results.
    \end{Verbatim}

    \subsection{The Same, but Fancy}\label{the-same-but-fancy}

The code below is almost exactly the same as in the example just above.
Observe that I added a style-sheet command (look for the region
hilighted by comments in the code).

You can have a standard style for your lab and agree on a given
style-sheet. Journals will often provide a \emph{MatPlotLib} style-sheet
file for you. Do you like it? Never appologize for your style (but
always be willing to change it to suit the whims of fashion.)

    \begin{tcolorbox}[breakable, size=fbox, boxrule=1pt, pad at break*=1mm,colback=cellbackground, colframe=cellborder]
\prompt{In}{incolor}{7}{\boxspacing}
\begin{Verbatim}[commandchars=\\\{\}]
\PY{k+kn}{import} \PY{n+nn}{matplotlib}\PY{n+nn}{.}\PY{n+nn}{pyplot} \PY{k}{as} \PY{n+nn}{plt}    \PY{c+c1}{\PYZsh{} plotting tools}
\PY{k+kn}{import} \PY{n+nn}{numpy} \PY{k}{as} \PY{n+nn}{np}                 \PY{c+c1}{\PYZsh{} we need the array object}
\PY{k+kn}{import} \PY{n+nn}{scipy}                       \PY{c+c1}{\PYZsh{} tools for science}

\PY{c+c1}{\PYZsh{}location\PYZus{}data = \PYZdq{}data/\PYZdq{}          \PYZsh{} Use either local folder or github folder. }
\PY{c+c1}{\PYZsh{}location\PYZus{}styles = \PYZdq{}styles/\PYZdq{}      \PYZsh{} Use github locations for Colab}
\PY{n}{location\PYZus{}data} \PY{o}{=} \PY{l+s+s2}{\PYZdq{}}\PY{l+s+s2}{https://raw.githubusercontent.com/blinkletter/PythonPresentation/main/data/}\PY{l+s+s2}{\PYZdq{}}
\PY{n}{location\PYZus{}styles} \PY{o}{=} \PY{l+s+s2}{\PYZdq{}}\PY{l+s+s2}{https://raw.githubusercontent.com/blinkletter/PythonPresentation/main/styles/}\PY{l+s+s2}{\PYZdq{}}

\PY{k}{def} \PY{n+nf}{MM}\PY{p}{(}\PY{n}{S}\PY{p}{,} \PY{n}{KM}\PY{p}{,} \PY{n}{Vmax}\PY{p}{)}\PY{p}{:}
    \PY{n}{rate} \PY{o}{=} \PY{n}{Vmax} \PY{o}{*} \PY{n}{S}\PY{o}{/}\PY{p}{(}\PY{n}{KM}\PY{o}{+}\PY{n}{S}\PY{p}{)}
    \PY{k}{return}\PY{p}{(}\PY{n}{rate}\PY{p}{)}

\PY{n}{conc} \PY{o}{=} \PY{p}{[}\PY{l+m+mf}{1.}\PY{p}{,}  \PY{l+m+mf}{2.}\PY{p}{,}  \PY{l+m+mf}{3.}\PY{p}{,}  \PY{l+m+mf}{5.}\PY{p}{,}  \PY{l+m+mf}{7.}\PY{p}{,}  \PY{l+m+mf}{9.}\PY{p}{,} \PY{l+m+mf}{10.}\PY{p}{,} \PY{l+m+mf}{15.}\PY{p}{]}          \PY{c+c1}{\PYZsh{} units are mM}
\PY{n}{rate} \PY{o}{=} \PY{p}{[}\PY{l+m+mf}{7.3}\PY{p}{,} \PY{l+m+mf}{20.7}\PY{p}{,} \PY{l+m+mf}{24.1}\PY{p}{,} \PY{l+m+mf}{34.3}\PY{p}{,} \PY{l+m+mf}{39.6}\PY{p}{,} \PY{l+m+mf}{48.2}\PY{p}{,} \PY{l+m+mf}{47.0}\PY{p}{,} \PY{l+m+mf}{55.2}\PY{p}{]}  \PY{c+c1}{\PYZsh{} units are uM/min}

\PY{c+c1}{\PYZsh{}\PYZsh{}\PYZsh{}\PYZsh{}\PYZsh{} MATH COMMANDS \PYZsh{}\PYZsh{}\PYZsh{}\PYZsh{}\PYZsh{}}

\PY{n}{conc} \PY{o}{=} \PY{n}{np}\PY{o}{.}\PY{n}{array}\PY{p}{(}\PY{n}{conc}\PY{p}{)} \PY{c+c1}{\PYZsh{} Convert lists to numpy arrays to enable math operations }
\PY{n}{rate} \PY{o}{=} \PY{n}{np}\PY{o}{.}\PY{n}{array}\PY{p}{(}\PY{n}{rate}\PY{p}{)} \PY{c+c1}{\PYZsh{}  that lists do not possess.}

\PY{c+c1}{\PYZsh{} calculate line fit and returns a list containing parameters}
\PY{n}{result} \PY{o}{=} \PY{n}{scipy}\PY{o}{.}\PY{n}{optimize}\PY{o}{.}\PY{n}{curve\PYZus{}fit}\PY{p}{(}\PY{n}{MM}\PY{p}{,} \PY{n}{conc}\PY{p}{,} \PY{n}{rate}\PY{p}{,} \PY{n}{p0} \PY{o}{=} \PY{p}{[}\PY{l+m+mi}{5}\PY{p}{,}\PY{l+m+mi}{70}\PY{p}{]}\PY{p}{)}  

\PY{c+c1}{\PYZsh{} extract the optimized parameter list (popt) and covariance matrix (pcov)}
\PY{p}{[}\PY{n}{popt}\PY{p}{,} \PY{n}{pcov}\PY{p}{]} \PY{o}{=} \PY{n}{result}  
\PY{p}{[}\PY{n}{KM}\PY{p}{,} \PY{n}{Vmax}\PY{p}{]} \PY{o}{=} \PY{n}{popt}      \PY{c+c1}{\PYZsh{} KM and Vmax were the two mitems i n the popt list}

\PY{n}{x\PYZus{}fit} \PY{o}{=} \PY{n}{np}\PY{o}{.}\PY{n}{linspace}\PY{p}{(}\PY{l+m+mi}{0}\PY{p}{,} \PY{n}{np}\PY{o}{.}\PY{n}{max}\PY{p}{(}\PY{n}{conc}\PY{p}{)}\PY{p}{,} \PY{l+m+mi}{100}\PY{p}{)} \PY{c+c1}{\PYZsh{} make x points for a smooth curve}
\PY{n}{y\PYZus{}fit} \PY{o}{=} \PY{n}{MM}\PY{p}{(}\PY{n}{x\PYZus{}fit}\PY{p}{,} \PY{n}{KM}\PY{p}{,} \PY{n}{Vmax}\PY{p}{)}    \PY{c+c1}{\PYZsh{} Calculate predicted curve using parameters}

\PY{c+c1}{\PYZsh{}\PYZsh{}\PYZsh{}\PYZsh{}\PYZsh{} PLOT COMMANDS \PYZsh{}\PYZsh{}\PYZsh{}\PYZsh{}\PYZsh{}}

\PY{c+c1}{\PYZsh{}\PYZsh{}\PYZsh{}\PYZsh{}\PYZsh{}\PYZsh{}\PYZsh{}\PYZsh{}\PYZsh{}\PYZsh{}THIS IS THE ONLY NEW CODE\PYZsh{}\PYZsh{}\PYZsh{}\PYZsh{}\PYZsh{}\PYZsh{}\PYZsh{}\PYZsh{}\PYZsh{}\PYZsh{}\PYZsh{}\PYZsh{}\PYZsh{}\PYZsh{}\PYZsh{}\PYZsh{}\PYZsh{}\PYZsh{}}
\PY{n}{plt}\PY{o}{.}\PY{n}{rcdefaults}\PY{p}{(}\PY{p}{)}   \PY{c+c1}{\PYZsh{} reset style to defaults}
\PY{n}{style} \PY{o}{=} \PY{l+s+s2}{\PYZdq{}}\PY{l+s+s2}{tufte.mplstyle}\PY{l+s+s2}{\PYZdq{}}
\PY{n}{plt}\PY{o}{.}\PY{n}{style}\PY{o}{.}\PY{n}{use}\PY{p}{(}\PY{n}{location\PYZus{}styles}\PY{o}{+}\PY{n}{style}\PY{p}{)}  \PY{c+c1}{\PYZsh{} apply style\PYZhy{}sheet file}
\PY{c+c1}{\PYZsh{}\PYZsh{}\PYZsh{}\PYZsh{}\PYZsh{}\PYZsh{}\PYZsh{}\PYZsh{}\PYZsh{}\PYZsh{}\PYZsh{}\PYZsh{}\PYZsh{}\PYZsh{}\PYZsh{}\PYZsh{}\PYZsh{}\PYZsh{}\PYZsh{}\PYZsh{}\PYZsh{}\PYZsh{}\PYZsh{}\PYZsh{}\PYZsh{}\PYZsh{}\PYZsh{}\PYZsh{}\PYZsh{}\PYZsh{}\PYZsh{}\PYZsh{}\PYZsh{}\PYZsh{}\PYZsh{}\PYZsh{}\PYZsh{}\PYZsh{}\PYZsh{}\PYZsh{}\PYZsh{}\PYZsh{}\PYZsh{}\PYZsh{}\PYZsh{}\PYZsh{}\PYZsh{}\PYZsh{}\PYZsh{}\PYZsh{}\PYZsh{}\PYZsh{}\PYZsh{}}

\PY{n}{plt}\PY{o}{.}\PY{n}{figure}\PY{p}{(}\PY{n}{figsize}\PY{o}{=}\PY{p}{(}\PY{l+m+mi}{4}\PY{p}{,}\PY{l+m+mi}{4}\PY{p}{)}\PY{p}{)}    \PY{c+c1}{\PYZsh{} create a new blank plot}

\PY{c+c1}{\PYZsh{} plot the LB line as red for comparison}
\PY{n}{plt}\PY{o}{.}\PY{n}{plot}\PY{p}{(}\PY{n}{x\PYZus{}fit}\PY{p}{,} \PY{n}{MM}\PY{p}{(}\PY{n}{x\PYZus{}fit}\PY{p}{,} \PY{n}{KM\PYZus{}lb}\PY{p}{,} \PY{n}{vmax\PYZus{}lb}\PY{p}{)}\PY{p}{,} \PY{l+s+s2}{\PYZdq{}}\PY{l+s+s2}{r\PYZhy{}}\PY{l+s+s2}{\PYZdq{}}\PY{p}{)}  
\PY{n}{plt}\PY{o}{.}\PY{n}{plot}\PY{p}{(}\PY{n}{conc}\PY{p}{,} \PY{n}{rate}\PY{p}{,} \PY{l+s+s2}{\PYZdq{}}\PY{l+s+s2}{ko}\PY{l+s+s2}{\PYZdq{}}\PY{p}{)}   \PY{c+c1}{\PYZsh{} plot the data as black \PYZdq{}k\PYZdq{} points \PYZdq{}o\PYZdq{}}
\PY{n}{plt}\PY{o}{.}\PY{n}{plot}\PY{p}{(}\PY{n}{x\PYZus{}fit}\PY{p}{,} \PY{n}{y\PYZus{}fit}\PY{p}{,} \PY{l+s+s2}{\PYZdq{}}\PY{l+s+s2}{k\PYZhy{}}\PY{l+s+s2}{\PYZdq{}}\PY{p}{)} \PY{c+c1}{\PYZsh{} plot the predicted line as black \PYZdq{}k\PYZdq{} line \PYZdq{}\PYZhy{}\PYZdq{}}

\PY{n}{plt}\PY{o}{.}\PY{n}{xlim}\PY{p}{(}\PY{l+m+mi}{0}\PY{p}{,}\PY{k+kc}{None}\PY{p}{)}          \PY{c+c1}{\PYZsh{} set axis limits so the plot starts at the origin}
\PY{n}{plt}\PY{o}{.}\PY{n}{ylim}\PY{p}{(}\PY{l+m+mi}{0}\PY{p}{,}\PY{l+m+mi}{80}\PY{p}{)}
\PY{n}{plt}\PY{o}{.}\PY{n}{xlabel}\PY{p}{(}\PY{l+s+sa}{r}\PY{l+s+s2}{\PYZdq{}}\PY{l+s+s2}{[S]}\PY{l+s+s2}{\PYZdq{}}\PY{p}{)}        \PY{c+c1}{\PYZsh{} label the two axis}
\PY{n}{plt}\PY{o}{.}\PY{n}{ylabel}\PY{p}{(}\PY{l+s+sa}{r}\PY{l+s+s2}{\PYZdq{}}\PY{l+s+s2}{\PYZdl{}}\PY{l+s+s2}{\PYZbs{}}\PY{l+s+s2}{nu\PYZdl{}}\PY{l+s+s2}{\PYZdq{}}\PY{p}{)}
\PY{n}{plt}\PY{o}{.}\PY{n}{title}\PY{p}{(}\PY{l+s+sa}{r}\PY{l+s+s2}{\PYZdq{}}\PY{l+s+s2}{Michaelis\PYZhy{}Menten plot}\PY{l+s+s2}{\PYZdq{}}\PY{p}{)}
\PY{n}{plt}\PY{o}{.}\PY{n}{tight\PYZus{}layout}\PY{p}{(}\PY{p}{)}        \PY{c+c1}{\PYZsh{} Prevents axis lables from falling off edge of plot}
\PY{n}{plt}\PY{o}{.}\PY{n}{savefig}\PY{p}{(}\PY{l+s+s2}{\PYZdq{}}\PY{l+s+s2}{plots/basics\PYZus{}plot4.pdf}\PY{l+s+s2}{\PYZdq{}}\PY{p}{)}   \PY{c+c1}{\PYZsh{} save the plot to this file}
\PY{n}{plt}\PY{o}{.}\PY{n}{show}\PY{p}{(}\PY{p}{)}                \PY{c+c1}{\PYZsh{} show the plot in this notebook and clear it}

\PY{n}{plt}\PY{o}{.}\PY{n}{rcdefaults}\PY{p}{(}\PY{p}{)}   \PY{c+c1}{\PYZsh{} reset style to defaults}

\PY{c+c1}{\PYZsh{}\PYZsh{}\PYZsh{}\PYZsh{}\PYZsh{} PRINT REPORT \PYZsh{}\PYZsh{}\PYZsh{}\PYZsh{}\PYZsh{}}

\PY{n+nb}{print}\PY{p}{(}\PY{l+s+sa}{f}\PY{l+s+s2}{\PYZdq{}}\PY{l+s+s2}{The Vmax is }\PY{l+s+si}{\PYZob{}}\PY{n}{Vmax}\PY{l+s+si}{:}\PY{l+s+s2}{0.4f}\PY{l+s+si}{\PYZcb{}}\PY{l+s+s2}{\PYZdq{}}\PY{p}{)} 
\PY{n+nb}{print}\PY{p}{(}\PY{l+s+sa}{f}\PY{l+s+s2}{\PYZdq{}}\PY{l+s+s2}{The KM is }\PY{l+s+si}{\PYZob{}}\PY{n}{KM}\PY{l+s+si}{:}\PY{l+s+s2}{0.4f}\PY{l+s+si}{\PYZcb{}}\PY{l+s+s2}{\PYZdq{}}\PY{p}{)} 
\PY{n+nb}{print}\PY{p}{(}\PY{l+s+s2}{\PYZdq{}}\PY{l+s+s2}{The red line is the predicted curve using Lineweaver\PYZhy{}Burke results.}\PY{l+s+s2}{\PYZdq{}}\PY{p}{)}
\end{Verbatim}
\end{tcolorbox}

    \begin{center}
    \adjustimage{max size={0.9\linewidth}{0.9\paperheight}}{01X_PlotBasics_files/01X_PlotBasics_12_0.png}
    \end{center}
    { \hspace*{\fill} \\}
    
    \begin{Verbatim}[commandchars=\\\{\}]
The Vmax is 81.0018
The KM is 6.8834
The red line is the predicted curve using Lineweaver-Burke results.
    \end{Verbatim}

    \subsection{What's Next}\label{whats-next}

There is much more to plotting and data analysis that \emph{Python} has
available. You will find it when you need it. Search and you shall find.
You will see examples of more advanced plotting in the next few
notebooks. If you like what you see, just steal it and change it to suit
your needs.


    % Add a bibliography block to the postdoc
    
    
    
\end{document}
